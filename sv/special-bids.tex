% Copyright 2007-2020 Joakim Nilsson
%
% This file is part of Combo Whist.
%
% Combo Whist is free software: you can redistribute it and/or modify
% it under the terms of the GNU General Public License as published by
% the Free Software Foundation, either version 3 of the License, or
% (at your option) any later version.
%
% Combo Whist is distributed in the hope that it will be useful,
% but WITHOUT ANY WARRANTY; without even the implied warranty of
% MERCHANTABILITY or FITNESS FOR A PARTICULAR PURPOSE.  See the
% GNU General Public License for more details.
%
% You should have received a copy of the GNU General Public License
% along with Combo Whist.  If not, see <http://www.gnu.org/licenses/>.

\newcommand{\nonTrump}{\textnormal{icke-trumf-bud}}

\begin{table}
	\caption{Specialbud}\label{tab:specialBids}
	\begin{center}
		\begin{tabularx}{\textwidth}{
			l
			S[table-number-alignment=center, table-format=1.0]
			S[table-number-alignment=center, table-format=1.0]
			p{3cm}
			|X
		}
			\textbf{Specialbud} &
			\rotccw{\textbf{Värde}} &
			\rotccw{\textbf{Ordning}} &
			\textbf{Inkompatibilitet} &
			\textbf{Tilläggsregler}
			\\[-3ex]

			\specialBidItem%
			{Triumf-Trumf}
			{-4}
			{1}
			{---}
			{%
				Spelföraren väljer ett valfritt kort innan spelet börjar. Detta kort blir \emph{triumf-trumfen}. Spelföraren bestämmer vem som tar sticket med triumf-trumfen när detta stick tas hem. Triumf-trumfen hamnar \emph{inte} i trumffärgen utan behåller sin gamla färg. Dessutom räknas triumf-trumfen---trots dess namn---inte heller som ett trumfkort.
			}

			\specialBidItem%
			{Lättja}
			{-3}
			{{---}}
			{---}
			{%
				För de stick därvar spelföraren inte spelar ut, spelar spelföraren sist.
			}

			\specialBidItem%
			{Potential}
			{-2}
			{{---}}
			{---}
			{%
				Om budet går hem markeras det med ett P, en \emph{potential}, i spelförarens kolumn. En spelare som innehar fler potentialer än en annan kan buda över den senares bud med ett bud värt lika mycket som det ledande budet.
			}

			\specialBidItem%
			{Start}
			{-2}
			{{---}}
			{---}
			{%
				Spelföraren spelar ut i första sticket.
			}

			\specialBidItem%
			{Brev}
			{-1}
			{2}
			{---}
			{%
				Före spelet börjar skickar alla spelare 3 kort i en riktning som spelföraren väljer (till höger, till vänster eller tvärs över).
			}

			\specialBidItem%
			{Järn}
			{-1}
			{{---}}
			{---}
			{%
				Essen rankas lägst istället för högst.
			}

			\specialBidItem%
			{Girighet}
			{0}
			{{---}}
			{---}
			{%
				I slutet av spelet läggs ett virtuellt stick till eller dras bort från spelföraren så att det missgynnar denne. Om budet går hem får spelföraren 1 extra-poäng.
			}

			\specialBidItem%
			{Ateljé}
			{1}
			{{---}}
			{Öppen Hand}
			{%
				Spelföraren väljer 4 kort som denne lägger i \emph{ateljén}. Dessa kort visas till samtliga spelare under spelets gång. Så fort det inte längre finns 4 kort i ateljen måste spelföraren, om möjligt, lägga dit ett nytt kort från handen.
			}

			\specialBidItem%
			{Mästarbrev}
			{{1 or 3}}
			{3}
			{---}
			{%
				Om budet budas i kombination med ett trumf-bud some inte är \emph{Grill} är värdet 3; annars 1. Före spelet börjar skickar alla utom spelföraren 3 kort till spelaren till höger (spelföraren hoppas över).
			}

			\specialBidItem%
			{Slut-Hund}
			{1}
			{{---}}
			{Noll}
			{%
				Spelföraren får inte ta hem det sista sticket.
			}

			\specialBidItem%
			{Öppen Trumf}
			{1}
			{{---}}
			{\nonTrump, Grill, Öppen Hand}
			{%
				Spelföraren måste spela med öppna trumfkort. Det vill säga, spelförarens trumfkort måste visas till samtliga spelare under spelets gång. Om detta bud kombineras med \emph{Ateljé} så får ateljén inte innehålla några trumfkort.
			}

			\specialBidItem%
			{Lås}
			{2}
			{{---}}
			{Noll}
			{%
				Spelföraren får inte ta hem något av de 3 första sticken.
			}

			\specialBidItem%
			{Pest}
			{2}
			{1}
			{Mästarskambud, Obesudlat Mästarspel, Noll, Skambud}
			{%
				Före alla andra händelser väljer spelföraren en färg som är \emph{pestfärgen}. Spelföraren får inte bli \emph{förpestad}; det vill säga, får inte ta hem enskilt flest kort (observera: \emph{ej} stick) i pestfärgen, såvida inte denne blir \emph{hedervärt förpestad} och tar hem hela pestfärgen samt att budet i övrigt går hem, i vilket fall spelföraren får 1 hedervärt extra-poäng. Spelföraren får inte spela ut i pestfärgen före pestfärgen har spelats på annat sätt, såvida inte spelföraren enbart besitter pest-kort.
			}

			\specialBidItem%
			{Straff}
			{2}
			{{---}}
			{---}
			{%
				Om budet inte går hem så dras 2 extra-poäng bort från spelförarens poängsumma.
			}

			\specialBidItem%
			{Öppen Hand}
			{3}
			{{---}}
			{Ateljé, Öppen Trumf}
			{%
				Spelföraren måste spela med öppen hand. Det vill säga, alla dennes kort måste visas för samtliga spelare under spelets gång.
			}
		\end{tabularx}
	\end{center}
\end{table}
