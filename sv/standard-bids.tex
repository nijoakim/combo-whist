% Copyright 2007-2019 Joakim Nilsson
%
% This file is part of Combo Whist.
%
% Combo Whist is free software: you can redistribute it and/or modify
% it under the terms of the GNU General Public License as published by
% the Free Software Foundation, either version 3 of the License, or
% (at your option) any later version.
%
% Combo Whist is distributed in the hope that it will be useful,
% but WITHOUT ANY WARRANTY; without even the implied warranty of
% MERCHANTABILITY or FITNESS FOR A PARTICULAR PURPOSE.  See the
% GNU General Public License for more details.
%
% You should have received a copy of the GNU General Public License
% along with Combo Whist.  If not, see <http://www.gnu.org/licenses/>.

\begin{table}
	\caption{Standardbud}\label{tab:standardBids}
	\begin{center}
		\begin{tabularx}{\textwidth}{
			l
			S[table-number-alignment=center, table-format=1.0]
			S[table-number-alignment=center, table-format=1.0]
			cc|X
		}
				\textbf{Namn} &
				\rotccw{\textbf{Värde}} &
				\rotccw{\textbf{Poäng}} &
				\rotccw{\textbf{Trumf}} &
				\rotccw{\textbf{Stick}} &
				\textbf{Tilläggsregler}
				\\[-3ex]

				\standardBidItem%
				{Skambud}
				{0}
				{1}
				{nej}
				{varierar}
				{%
					Spelföraren får inte ta hem flest stick---inte ens om nån annan spelare har tagit hem lika många stick.
				}

				\standardBidItem%
				{Trumf}
				{1}
				{1}
				{ja}
				{min 5}
				{%
					Spelföraren bestämmer trumffärg.
				}

				\standardBidItem%
				{Ungefär}
				{1}
				{1}
				{nej}
				{varierar}
				{%
					Före spelets början gissar spelföraren på två möjliga mängder stick denne kommer att ta hem. Spelföraren måste ta hem en av de möjliga mängderna stick som gissades.
				}

				\standardBidItem%
				{Grill}
				{1}
				{2}
				{ja}
				{min 5}
				{%
					Spelföraren börjar med att bestämma trumffärg. Denna trumffärg gäller bara första sticket. Därefter blir den färg som spelades ut i föregångde stick ny trumffärg och så fortsätter det till spelets slut.
				}

				\standardBidItem%
				{Limbo}
				{2}
				{1}
				{nej}
				{varierar}
				{%
					Spelföraren måste ta hem färre stick under sista halvan av spelet än under första. Eftersom de två halvorna inte kan göras lika stora är den sista halvan den kortare.
				}
				
				\standardBidItem%
				{Spel}
				{2}
				{2}
				{nej}
				{min 5}
				{%
					---
				}

				\standardBidItem%
				{Exakt}
				{3}
				{2}
				{nej}
				{varierar}
				{%
					Före spelets början gissar spelföraren på en möjlig mängd stick denne kommer att ta hem. Spelföraren måste ta hem den mängd stick som gissades.
				}

				\standardBidItem%
				{Mästarskambud}
				{3}
				{2}
				{nej}
				{varierar}
				{%
					Spelföraren måste ta hem färst stick. Om ingen tar hem färre stick än spelföraren går budet hem.
				}

				\standardBidItem%
				{Maxtrumf}
				{3}
				{3}
				{ja}
				{min 7}
				{%
					Spelföraren väljer trumffärg.
				}

				\standardBidItem%
				{Smygtrumf}
				{3}
				{3}
				{ja}
				{min 5}
				{%
					Spelföraren väljer trumffärg. Denne får dock inte välja en trumffärg som denne har flest kort i.
				}

				\standardBidItem%
				{Mästarspel}
				{4}
				{3}
				{nej}
				{varierar}
				{%
					Spelföraren måste ta hem enskilt flest stick.
				}

				\standardBidItem%
				{Noll}
				{4}
				{4}
				{nej}
				{0}
				{%
					---
				}

				\standardBidItem%
				{Mästartrumf}
				{6}
				{6}
				{ja}
				{min 5}
				{%
					Spelaren till vänster om spelföraren bestämmer trumffärg, men först får de andra icke-spelförarna säga vilken trumffärg de föredrar och hur mycket de föredrar denna på en skala från 1 till 5 (utan motivering).
				}

				\standardBidItem%
				{Obesudlat Mästarspel}
				{9}
				{$x$}
				{nej}
				{13}
				{%
					Om budet går hem får spelföraren lika många poäng som kombinations-budets värde. Skulle dessutom kombinations-budets värde vara 13 eller högre vinner spelföraren spelet omedelbart oavsett poängställning. När detta inträffar erhåller dessutom spelföraren rätten att titulera sig \emph{Obesudlad~Mästare~av~Kombinations-Whist} under resten av sitt liv.
				}
		\end{tabularx}
	\end{center}
\end{table}
